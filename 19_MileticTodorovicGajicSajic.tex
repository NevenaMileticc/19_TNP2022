% !TEX encoding = UTF-8 Unicode

\documentclass[a4paper]{article}

\usepackage{color}
\usepackage{url}
\usepackage[T2A]{fontenc} % enable Cyrillic fonts
\usepackage[utf8]{inputenc} % make weird characters work
\usepackage{graphicx}

\usepackage[english,serbian]{babel}
%\usepackage[english,serbianc]{babel} %uključiti babel sa ovim opcijama, umesto gornjim, ukoliko se koristi ćirilića

\usepackage[unicode]{hyperref}
\hypersetup{colorlinks,citecolor=green,filecolor=green,linkcolor=blue,urlcolor=blue}

%\newtheorem{primer}{Primer}[section] %ćirilični primer
\newtheorem{primer}{Primer}[section]

\begin{document}

\title{3D štampa\\ \small{Seminarski rad u okviru kursa\\Tehničko i naučno pisanje\\ Matematički fakultet}}


\author{    Nevena Miletić\\ mi22193@alas.matf.bf.ac.rs
        \and Matija Todorović\\ mi22282@alas.matf.bf.ac.rs
        \and Đorđe Sajić\\ mi22237@alas.matf.bf.ac.rs
        \and Relja Gajić\\ mi22131@alas.matf.bf.ac.rs
 }

\date{10.novembar 2022.}
\maketitle
\abstract{
Aditivna proizvodnja je moderna tehnologija proizvodnje trodimenzionalnih objekata. 3D štampa predstavlja generalno brže, jeftinije i lakše rešenje od drugih tehnologija proizvodnje 3D objekata. Omogućava izradu maketa delova i sklopova od više različitih materijala, različitih mehaničkih i fizičkih svojstava u jedinstvenom procesu. Ova tehnologija proizvodi modele koji verno oponašaju izgled, utisak i funkcionalnost proizvoda prototipa. U poslednjih nekoliko godina 3D štampači su postali finansijski dostupni malim i srednjim preduzećima, čime se izrada prototipa pomera iz teške industrije i u kancelarijsko okruženje.Iako su pomenute
mašine do skora bile dostupne samo malom broju ljudi, uglavnom naučnicima, to više nije slučaj. Cena 3D štampača su sve niže.
\tableofcontents

\newpage

\section{Uvod}
\label{sec:uvod}
Aditivna proizvodnja predstavlja način proizvodnje gde se iz digitalnog modela stvara trodimenzionalan predmet. Kako to zapravo izgleda?
\bigbreak

3D štampa može da se zamisli kao štampa tankih horizontalnih slojeva. Ti slojevi predstavljaju horizontalni presek predmeta.
\bigbreak Tehnologija je nastala 80-ih godina prošlog veka i koristila se pre svega za proizvodnju prototipova. Ipak, od tada 3D štampa počinje da se koristi kao tehnika za stvaranje proizvoda gotovo u svim industrijama.
Proizvodi koji se 3d štampaju mogu biti od različitih materijala, od plastike, preko keramike, do metala. Tehnologija se i dalje razvija a novi materijali pomoću kojih se štampa se uvode neverovatnom brzinom. Sada je moguće i istovremeno uklapanje različitih vrsta materijala. Osim izrade prototipa, 3D štampači nude veliki potencijal za proizvodnju različitih aplikacija u oblasti proizvodnje nakita, obuće, industrijskog dizajna, arhitekture, automobilske industrije, avio, stomatološke i medicinske industrije.

\section{Istorija razvoja 3D štampe}


\bigbreak Možda ste čuli za 3D štampu koja je proglašena budućnošću proizvodnje. I sa načinom na koji se tehnologija razvija i širi komercijalno, može se vrlo dobro nadoknaditi u oprezu koja ga okružuje. Šta je 3D štampanje? A ko je to izmislio? 
\bigbreak Najbolji primer koji opisuje kako rade 3d štampe dolaze iz TV serije Star Trek: The Next Generation. U tom fiktivnom futurističkom univerzumu, posada na svemirskog broda koristi mali uređaj koji se zove replikator za stvaranje skoro bilo čega, od igračaka čak do hrane i pića. 
\bigbreak Iako obe "mašine" konstruišu trodimenzionalne objekte 3D štampanje nije skoro toliko sofisticirano. Pošto replikator manipuliše subatomske čestiče kako bi proizveo bilo koji objekat, 3D štampači "štampaju" materijale u sukcesivnim slojevima kako bi formirali objekat.  
\bigbreak Istorijski gledano, razvoj tehnologije počinje početkom 1980-ih godina, čak i pre pomenute TV emisije. 1981. godine, Hideo Kodama iz Instituta za industrijska istraživanja Nagoya prvi je objavio račun o tome kako se materijali koji se nazivaju fotopolimeri koji su ojačani prilikom izlaganja UV zračenju mogli iskoristiti za brzo izradu čvrstih prototipa. Iako je njegov članak postavio temelje za 3D štampanje, on nije prvi napravio 3D štampač. 
\bigbreak Ta prestižna čast ide inženjeru Chuck Hull-a, koji je dizajnirao i napravio prvi 3D štampač 1984.On je radio za kompaniju koja je koristila UV lampe za oblikovanje čvrstih, izdržljivih premaza za stolove kada je dobio ideju da iskoristi ultraljubičastu tehnologija za izradu malih prototipa. 
\bigbreak Ključ za izradu takvog štampača bili su fotopolimeri koji su ostali u tečnom stanju sve dok nisu reagovali na ultraljubičastu svetlost. Sistem koji je Hull na kraju razvio, poznat kao je stereolitografija, koristio je UV zračenje kako bi skicirao oblik objekta iz kadra tečnih fotopolimera. 
\bigbreak On je patentirao tehnologiju 1984. godine, ali je tri nedelje nakon što je tim francuskih pronalazača Alain Le Méhauté, Olivier de Witte i Jean Claude André podneo patent za sličan proces. Međutim, njihovi poslodavci odlucili su da prestanu da dalje razvijaju tehnologiju zbog "nedostatka poslovne perspektive". To je omogućilo Hullu da autorsko pravo pod pojmom "Stereolithography". Njegov patent, naslovljen "Aparat za proizvodnju trodimenzionalnih objekata stereolitografijom" objavljen je u martu 11, 1986. Te godine, Hull je takođe formirao 3D sisteme u Valensiji, Kalifornija.
Dok je Hullov patent obuhvatao mnoge aspekte 3D štampanja, uključujući dizajn i operativni softver, tehnike i razne materijale, drugi izumitelji bi se nadovezali na koncept sa različitim pristupima. 1989. godine dodeljen je patent Carlu Deckardu, diplomantu Univerziteta u Teksasu koji je razvio metod pod nazivom selektivno lasersko sinterovanje. Sa SLS-om, laserski zrak se koristi za prilagođavanje praškastih materijala, kao što je metal, zajedno kako bi se formirao sloj objekta. 
\bigbreak Svježi prah bi se dodao na površinu nakon svakog uzastopnog sloja. Ostale varijacije kao što su direktno metalno lasersko sinterovanje i selektivno lasersko taljenje se takođe koriste za izradu metalnih predmeta. 
\bigbreak Najpopularniji i najprepoznatljiviji oblik 3D štampanja naziva se modeliranje spojenih depozita. FDP, koji je razvio pronalazač S. Scott Crump, postavlja materijal u slojeve direktno na platformu. Materijal, obično smola, se isprazni kroz metalnu žicu i, nakon puštanja kroz mlaznicu, odmah se ojača. Ideja je došla u Crump 1988. godine, dok je pokušavao da napravi igračku žabu za svoju ćerku izdavanjem voska za sveće pomoću pištolja za lepak. 
\bigbreak 1989. godine Crump je patentirao tehnologiju i sa svojom suprugom osnovao Stratasys Ltd. za proizvodnju i prodaju 3D štamparskih mašina za brzo prototipiranje ili komercijalnu proizvodnju. 
Kompaniju su izveli u javnost 1994. godine, a do 2003. godine FDP je postao najprodavanija tehnologija brže prototipizacije. 

\section{Tehnike 3D štampe}
\label{sec:naslov1}
Prema tehnici 3D štampanja možemo razlikovati sledeće tehnologije:
\begin{itemize}
\item \textbf{Inkjet}
\item \textbf{Fused Deposition Modelin(FDM)}
\item \textbf{Stereolitografija}
\item \textbf{Selektivno lasersko sinterovanje (SLS)}
\item \textbf{Proizvodnja objekata laminacijom (LOM)}
\end{itemize} 



 


\subsection{Inkjet}
\label{subsec:podnaslov1}
Ubrizgivanje (eng, Inject) - kreira protorip na osnovu 3D modela tako što pravi sloj po sloj projekta.
\bigbreak Ističe se po atributu da je adaptivan na razne oblike tečnih materijala što pruža stvaranje konduktivnih ili izolacionih struktura u visokoj rezoluciji. 
\bigbreak Za razliku od drugih procesa, injekt štampanju nije potrebno dodatno prerađivanje i obrada nako izvršenog štampanja.
\bigbreak Prilikom izrade materijali u tečnom stanju se stavljaju na vrh za štampanje (eng, Printing head) koji pravi dati projekat.
\bigbreak Moze uz glavni materija koji se koristi za model da se doda pomoćni materijal (eng, Support material) koji mu daje čvrstinu i stabilnos prilikom izrade.
\begin{center}
\includegraphics[width=.5\textwidth ]{Tehnikeslike/Inject.png}
\end{center}

\subsection{Fused Deposition Modelin(FDM)}
\label{subsec:podnaslov2}
FDM (Fused Deposition Modeling) - predstavlja najkorišćenu i najjeftiniju tehniku za 3D štampanje. Lak je za korišćenje, on koristi termoplastične filamente sa ektruzijom između svakog sloja.
\bigbreak Najčešće se koristi u developovanju proizvoda  i test modela koji inžinjeri koriste da provere da li oblik odgovara datom uređaju po dimenzijama. 
\bigbreak Sastoji se od podloge na kojoj se izvršava štampanje, displeja za štampanje, mašine koja zagreva filamente i kulera koji održava temperaturu uređaja.

\begin{center}
\includegraphics[width=.5\textwidth ]{Tehnikeslike/FDM.PNG}
\end{center}

\subsection{Stereolitografija}
\label{subsec:podnaslov3}
Stereolitografija (Stereolithography) -poznatija kao ŠLA, koristi rezervoar tečnog fotopolimera koji formira sloj po sloj projekta , gde se nakon formiranja svakog sloja polimer stvrdne uz pomoć ultraljubičastog svetla.
\bigbreak Nakon svakog odrađenog sloja podloga se spušta kako bi krenula obradu sledećeg.Kada se završi obrada 3D modela potrebno je izvršeni projekat oprati rastvaračem da bi se uklonila tečnos opasna po život.
\bigbreak Ova tehnika se najčešće koristi za kreiranje 3D modela visokih rezolucija.  Stereolitografija koristi pomoćne stubove koji sprečavaju da dođe do deformacije prilikom izgradnje sledećih slojeva koji se nalaze iznad njega.

\begin{center}
\includegraphics[width=.5\textwidth ]{Tehnikeslike/Stereolitografija.PNG}
\end{center}

\bigbreak
\bigbreak

\subsection{Selektivno lasersko sinterovanje (SLS)}
\label{subsec:podnaslov4}
Izborno lasersko lemljenje(eng, Selective Laser Sintering (SLS)) - koristi se za pravljenje prototipa od metala i plastike. Uz pomoć zrna materijala (eng, powder beds) mašina pravi dizajnove sloj po sloj, pri čemu koristi lasersko grejanje i zbijanje (sinterovanje) praha datog materijala. 
\bigbreak Nakon svakog sloja postolje se spušta i dodaje novi sloj materijala sve dok mašina ne završi izgradnju modela. Pošto čvrstina modela zavisi od najveće temperature mašine SLS zagreje prevremeno prah materijala ispod njengove temperature topljenja kako bi dostigao maksimalne temperature za najmanji period, što isto omogućava da laser lakše izreže željene oblike za sloj. 
\bigbreak Kao što je pokazano na slici rezervoari (eng, powder tank) stavljaju prah na vrh prethodnog sloja, dok se njegova platforma spušta za sledeći sloj.
\bigbreak Sečivo za slojeve(recoater blade) prolazi i ravnja vrh modela, nakon toga dolazi do građenja oblika koji radi laser. 
\begin{center}
\includegraphics[width=.5\textwidth ]{Tehnikeslike/Sls.png}
\end{center}
\newpage
\subsection{Proizvodnja objekata laminacijom (LOM)}
\label{subsec:podnaslov5}
Proizvodnja objekata laminacijom (eng, Laminated object manufacturing (LOM)) - Korišćenjem papira premazanim lepkom, plastiku ili metalni laminat kao sredstvo za 3D štampanje. 
\bigbreak Hartije se lepe jedna za drugu sloj po sloj na platformi uz pomoć motora koji na krajevima imaju rolne korišćenog materijala (Material spool).
\bigbreak Na radnoj površini se seku u željeni oblik pomoću noža ili lasera, a neiskorišćeni ostatak idu u rolnu koja se nalazi u smeru obrtanja motora.
\bigbreak Nakon tog procesa one se mogu dodatno doraditi šmirglanjem i bušenjem ako je potrebno.

\begin{center}
\includegraphics[width=.5\textwidth ]{Tehnikeslike/LOM.jpg}
\end{center}

\begin{center}
\begin{tabular}{||c c c ||} 
 \hline
 Broj & Tip  & Materijal \\ [1ex] 
 \hline\hline
 1 & Tečni materijali \\ 
 \hline
 2 & FDM & termoplastični filamenti \\
 \hline
 3 & SLA & tečni fotopolimer\\
 \hline
 4 & SLS & Metal ili plastika\\
 \hline
 5 & LOM & papir premazan lepkom, plastiku ili metalni laminat\\ [1ex] 
 \hline
\end{tabular}
\end{center}

\newpage

\section{Primena}
\label{sec:Primena}

U proteklih nekoliko godina, štampači postaju sve dostupniji i dostupniji. Više se ne koriste isključivo u teškoj industriji, već i u manjim preduzećima i čak i za ličnu upotrebu. Koriste se u najrazličitije svrhe i skoro svim granama privrede. Kao što je već napomenuto, broj raznovrsnih materijala koji se koriste za rad sa 3d štampačima je ogroman i raste vrtoglavom brzinom. To omogućava veliku raznovrsnost njihove primene. 

\subsection{Medicina}
\label{subsec:podnaslov6}

Primena 3d štampača u medicini je počela 2014. godine u Velikoj Britaniji kada je pacijentu koji je od posledica raka izgubio gotovo pola kože lica, upravo putem 3d štampe napravljen novi deo lica i postavljen na mesto starog. Ovo predstavlja veliki korak u modernoj medicini jer se prvi put dogodilo da je čovek u mogućnosti da odštampa neki deo svog tela. 
\bigbreak Ovakvi procesi se izvršavaju primenom fotogrametrije. Proces se sastoji iz niza kamera koje prave fotografije zdravih tkiva pacijenta, a zatim ga računarski sklapaju u jednu 3d celinu koja će kasnije biti odštampana. Proces slikanja i izrade celine je vrlo brz i traje svega nekoliko sati, dok je proces štampanja vrlo delikatan i sa veoma malim prostorom za grešku, naravno, traje znatno duže od prethodnih. Pričvršćivanje novog, štampanog tkiva se izvršava hirurškim zahvatom.

\subsection{Prototipovi}
\label{subsec:podnaslov7}

Ovo je najvažniji i najrasprostranjeniji vid primene 3d štampača. Dizajnerima je znatno olakšan i ubrzan rad. Štampač može uz korišćenje male količine materijala i u kratkom vremenskom roku da stvori prototip za čiju bi izradu bilo potrebno značajno više. Ovaj vid primene se koristi u gotovo svim granama privrede.

\subsection{Vojna primena}
\label{subsec:podnaslov8}

Naravno, ljudi su vrlo brzo našli primenu 3d štampačima u vojnoj industriji. Konkretno, za izradu delova za pištolje i puške, kao i za izradu bojeve municije. Počelo je tako što je jedan mladi amerikanac došao na ideju da napravi jeftine delove za svoju pušku, što mu je i pošlo za rukom. Ono što je izuzetno privlačno kompanijama za proizvodnju oružja, kao i vladama i organizacijama koje ga kupuju, je upravo mogućnost pravljenja oružja kome se ne može ući u trag. U vojnoj industriji se očekuje najveći porast primene štampača. 

\subsection{Izrada odeće i obuće}
\label{subsec:podnaslov9}

Razne kompanije za proizvodnju odeće i obuće obilato koriste 3d štampače. Ne toliko za pravljenje konkretnih celina kao što su patike, već za pravljenje raznih manjih delova. Poznati su upravo delovi za sportsku odeću poput krampona za kopačke ili ojačanih vrhova pertli.

\section{Zaključak}
\label{sec:Zaključak}

U proteklih nekoliko godina, primena kao i dostupnost 3d štampanja raste, 3d štampa prelazi iz sveta fikcije u realan svet i uskoro ce se pretopiti u savremenu upotrebu. Od krucijalne važnosti je dalji razvitak ove tehnologije jer su, teorijski, mogućnosti neograničene i trenutno je jedini limit nivo tehnologije našeg vremena. 3d štampanje će moći imati ogroman udeo u raznim sverama kao što su medicinske svrhe, inženjering, avio i automobilske industrije...
\newpage
 
\section{Literatura}
\label{sec:Literatura}

https://3dstampa.rs/sta-je-3d-stampa/
https://3dstampa.rs/sta-sve-moze-da-se-3d-stampa/
https://bs.eferrit.com/ko-je-pronasao-3d-stampanje/
https://www.rapiddirect.com/blog/3d-prototyping/
https://www.nano-di.com/resources/blog/2019-the-3d-inkjet-printing-process-explained
https://www.twi-global.com/technical-knowledge/faqs/what-is-laminated-object-manufacturing-lom
https://tractus3d.com/knowledge/learn-3d-printing/fdm-3d-printing/

\end{document}
